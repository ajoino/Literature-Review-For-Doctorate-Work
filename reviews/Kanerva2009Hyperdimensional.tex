\begin{review}{Kanerva2009Hyperdimensional}
                {Holographic reduced represantation\\
                 Holistic record\\
                 Holistic mapping\\
                 Random indexing\\
                 Cognitive code\\
                 von Neumann architecture}
                {
                    The 1990s saw the emergence of cognitive models that depend on very high dimensionality and randomness. They include Holographic Reduced Representations, Spatter Code, Semantic Vectors, Latent Semantic Analysis, Context-Dependent Thinning, and Vector-Symbolic Architecture. They represent things in highdimensional vectors that are manipulated by operations that produce new high-dimensional vectors in the style of traditional computing, in what is called here hyperdimensional computing on account of the very high dimensionality. The paper presents the main ideas behind these models, written as a tutorial essay in hopes of making the ideas accessible and even provocative. A sketch of how we have arrived at these models, with references and pointers to further reading, is given at the end. The thesis of the paper is that hyperdimensional representation has much to offer to students of cognitive science, theoretical neuroscience, computer science and engineering, and mathematics. © 2009 Springer Science+Business Media, LLC.
                }
    This paper addresses the incompatibility of normal computing with the computing of the human brain with a simple observation: no two human brains are the same, but they can still do the same things.
    In fact, the way a brain stores and uses knowledge is somewhat random.
    The author suggests a new kind of computer architecture (unnamed, which is good because in my mind it's not really a complete description) that uses \emph{hyperdimensional} (cool buzzword there fella) vectors.
    
    Hyperdimensional vectors are just regular vectors but the author defines them as having in the order of a thousand or more elements.
    Computing with hyperdimensional vectors differs in one main way: Representation.
    In normal computers, a concept is (usually) represented by a unique bitstring.
    An example would be the color red using RGB encoding, it is 0xFF0000.
    Any bitstring (interpreted as a color) with a value different from 0xFF0000 would not be interpeted as red, or at least not the same red.
    
    When hyperdimensional vectors are used, a holistic (or holographic) mapping can be used instead.
    Using the same example as before, in holistic mapping all vectors that are sufficiently similiar are interpreted as red, like 0xFD0000 or 0xFF0101 for example.
    The paper then goes into detail on how memory and different concepts and relations can be mapped to these hyperdimensional vectors.
    Some examples of when it works out and when it doesn't are also presented.
    A very intersting thing to note is that this 'architecture' is random by its nature.
    There is never a guarantee that things will work, just a very high probability.
    This is a property that it shares with the human mind, where two different people often find the same solution to a problem despite being very different.
    
    The author also stresses that this is not a completed story by any means.
    He presents many different solutions using hyperdimensional vectors and basically says that the hardware we have today is not suitable for these kind of vectors.
    And that is really the big disadvantage of this model in my eyes.
    It is a very different way of designing a computer than what we are used to and specialized hardware is probably the way to go.
    
    I think this paper is fantastic in its layout.
    The author goes a very different route than most by not citing any previous works before the second to last section.
    This is despite there being places where he probably should have done it.
    But this means that there is really nothing to distract you from the core idea, that holistic mapping is a really good way to deal with a bunch of problems using simple calculations.
    
    I would really like to use something like this in my work, it seems really fun and interesting.

\end{review}
