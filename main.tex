\documentclass[10pt]{article}
\usepackage[utf8]{inputenc}
\usepackage[english]{babel}
 
\usepackage{blindtext}
\usepackage{datetime2}
\usepackage{verbatim}
\usepackage[backend=biber]{biblatex}
\DeclareNameAlias{labelname}{family-given}
\addbibresource{jacnilLiteratureReview.bib}
\usepackage{hyperref}
\hypersetup{
    colorlinks,
    citecolor=black,
    filecolor=black,
    linkcolor=black,
    urlcolor=black
}

\usepackage{geometry}
 \geometry{
    left=10mm,
    top=20mm,
    right=10mm,
    bottom=20mm
 }

\usepackage{Review}
 
\title{Literature Review}
\author{Jacob Nilsson}
\date{Last updated \today}

\begin{document}
 
\maketitle
\vfill
\centering
\begin{minipage}[p]{0.75\textwidth}

    \section*{\LARGE Preface}
        This document is a review of articles I have found interesting and related to my doctorate work at Luleå University of Technology.
        This is a living document and reviews will be added and changed over time.
        This document is intended to help my work and to share with colleges all over the world.
        While my ambition is to keep this document as orderly as possible, knowing myself the quality will probably deteriorate over time.
        If that happens, the reader may email me at \verb|jacob.nilsson@ltu.se| to kindly remind me to get my shit together.\\[1em]
        Regards\\[0.3em]\phantom{OO}\textit{Jacob Nilsson}\\\phantom{OO}\footnotesize 2017-07-04
    
\end{minipage}
\vfill
\newpage
\begin{minipage}[p]{0.75\textwidth}
\section*{\LARGE Review structure}
    Each review consists of two parts: the information part and the review part.
    The information part contains the following:
    \begin{enumerate}
        \item Title,
        \item Authors,
        \item Year published,
        \item Keywords,
        \item Abstract,
        \item Bibliography entry -- For use with my jacnil\_literature\_review.bib file (must be renamed!).
    \end{enumerate}
    The review part contains my written review of the article.
    It is written as a normal text and addresses the following points:
    \begin{enumerate}
        \item Overall context,
        \item Key Findings,
        \item Essential details,
        \item Important weaknesses,
        \item Possible future work,
        \item My thoughts on the paper and the possibilities it presents.
    \end{enumerate}
    The points will be addressed roughly in that order but not necessarily.
    I'm also considering a quick reference part where these questions are answered in a list format.    
    \vspace{0.3cm}\\
    \-\phantom{OO}\textit{2017-07-05}
\end{minipage}
\newpage
%\section*{\citetitle{Turney2010VsmOverview}}
%\cite{Turney2010VsmOverview}
%\citetitle{Turney2010VsmOverview}
%\Large\citeauthor{Turney2010VsmOverview}
%\citefield{Turney2010VsmOverview}{journaltitle}
%\citefield{Turney2010VsmOverview}{type}
\flushleft
\begin{review}
        {Turney2010VsmOverview}
        {N/A}
        {
            Computers understand very little of the meaning of human language. 
            This profoundly limits our ability to give instructions to computers, the ability of computers to explain their actions to us, and the ability of computers to analyse and process text. 
            Vector space models (VSMs) of semantics are beginning to address these limits. 
            This paper surveys the use of VSMs for semantic processing of text. 
            We organize the literature on VSMs according to the structure of the matrix in a VSM. 
            There are currently three broad classes of VSMs, based on term-document, word-context, and pair-pattern matrices, yielding three classes of applications. 
            We survey a broad range of applications in these three categories and we take a detailed look at a specific open source project in each category. 
            Our goal in this survey is to show the breadth of applications of VSMs for semantics, to provide a new perspective on VSMs for those who are already familiar with the area, and to provide pointers into the literature for those who are less familiar with the field. 
            © 2010 AI Access Foundation and National Research Council Canada.
        }
    This paper tries to give a survey of the use of \emph{Vector Space Models} (VSM) in semantics research.
    Their motivation is that there had been no study of the field as a whole.
    They present a new framework for VSMs, i.e. they try to make the terminology used in the field consistent.
    They also present some recent developments that had not been covered but other surveys of the field (These developments were made by Turney).
    The applications they present are from natural language processing and computational linguistic.
    
    The paper, being a survey, does not present any new results.
    The big contribution is their attempt to consolidate the terminology used in the field.
    These are the \emph{term-document} matrix (for finding similarities between documents), the \emph{word-context} matrix (for finding \emph{attributional similarities} between words) and the \emph{pair-pattern} matrix (for finding \emph{relational similarities} between words).
    Attributionally similar words are those that share many features (like doctor and nurse) and relationally similar words are words whose relations have a lot in common (like carpenter:wood and mason:stone).
    The paper is fairly long and contains a lot examples and goes into detail how the VSMs are constructed, the linguistic and mathematical pre-processing that needs to done and how the different matricies can then be used.
    They are fully aware that the matrix methods are fairly slow, but also present new research (random indexing) that can improve the computational speed.
    
    This paper doesn't have many weaknesses but one I'd like to point out is that it only (as it says on the cover) looks into methods using VSMs.
    There are other methods that can be used and they use a page to briefly present them.
    However, considering the nature of the paper that is not really a problem.
    They also answer the question of word order in VSMs by presenting both possible and actual solution to the problem.
    The main open question they present is if these statistical models are enough to figure out what people mean?
    
    I think this paper is brilliant as an introduction to VSMs.
    It presents seemingly all views of the subject and I'm struggling to find bad things to say about it.
    Perhaps I will find some when I learn more about the subject.
    I really like the idea of VSMs, though considering that I will work on computers that people claim are 'less random' than humans, perhaps there are some simplifications that can be done?
    Also, random indexing seems like the way to go!
\end{review}

\begin{review}
        {Turney2006Similarity}
        {N/A}
        {
        There are at least two kinds of similarity. 
        \textbf{Relational similarity} is correspondence between relations, in contrast with \textbf{attributional} similarity, which is correspondence between attributes.
        When two words have a high degree of attributional similarity, we call them \textbf{synonyms}. 
        When two pairs of words have a high degree of relational similarity, we say that their relations are \textbf{analogous}. 
        For example, the word pair mason:stone is analogous to the pair carpenter:wood. 
        This article introduces Latent Relational Analysis (LRA), a method for measuring relational similarity. LRA has potential applications in many areas, including information extraction, word sense disambiguation, and information retrieval. 
        Recently the Vector Space Model (VSM) of information retrieval has been adapted to measuring relational similarity, achieving a score of 47\% on a collection of 374 college-level multiple-choice word analogy questions. 
        In the VSM approach, the relation between a pair of words is characterized by a vector of frequencies of predefined patterns in a large corpus. 
        LRA extends the VSM approach in three ways: (1) The patterns are derived automatically from the corpus, (2) the Singular Value Decomposition (SVD) is used to smooth the frequency data, and (3) automatically generated synonyms are used to explore variations of the word pairs. 
        LRA achieves 56\% on the 374 analogy questions, statistically equivalent to the average human score of 57\%. 
        On the related problem of classifying semantic relations, LRA achieves similar gains over the VSM. 
        © 2006 Association for Computational Linguistics.
        }
    This paper describes a method for measuring relational similarity (analogies) using a method the author calls \emph{Latent Semantic Analysis} (LRA).
    LRA is a VSM approach, using the pair-pattern matrix as described in \cite{Turney2010VsmOverview}.
    
    The first few sections are spent describing the difference between attributional and relational similarity, what the attributional similarity has been useful for and what relational similiarity can be useful for.
    The suggestions are improvement of the structure-mapping engine \cite{Gentner1983Structure}, improved metaphor recognition, question answering, automatic thesaurus generation etc.
    
    The LRA algorithm is introduced by first describing VSMs in general and then the specifics of the LRA.
    The important specifics are that LRA uses a pair-pattern matrix.
    For each pair it can find, it uses a thesaurus to create alternate pairs, so if the original pattern is $A : B$, it also creates the pairs $A' : B$ and $A : B'$.\cite{Kanerva2009Hyperdimensional}
\end{review}

\begin{review}{Kanerva2009Hyperdimensional}{}{}

\end{review}

\begin{review}{Sandin2017RandomIndexing}
    test igen
\end{review}

\listofreview

\newpage
\printbibliography

\end{document}